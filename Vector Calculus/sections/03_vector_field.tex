\documentclass[../main.tex]{subfiles}
\graphicspath{{\subfix{../media/}}}


\begin{document}
	
	\begin{definition}
		A vector field on $\mathbb{R}^n$ is a function $\mathbf{F} : D \subset \mathbb{R}^n \rightarrow \mathbb{R}^n$ that assign to each point $\mathbf{x}$ in domain\footnote{domain could be a subset of $\mathbb{R}^n$ or all of it} $D$ a vector $\mathbf{F}(\mathbf{x}) \equiv \langle x_1, \hdots, x_n \rangle \in \mathbb{R}^n$
	\end{definition}
	
	\par For visualisation of vector fields defined in $\mathbb{R}^2$ or $\mathbb{R}^3$ one would plot some vectors samples across the domain to avoid clutter.  See plots of vector fields in $\mathbb{R}^2$ or $\mathbb{R}^3$ receptively.

	\begin{figure*}[h]
		\centering
		\begin{subfigure}{0.47\textwidth}
			\centering
			\begin{tikzpicture}
				\begin{axis}[
				  	view={0}{90}, % for a view 'from above',
				  	domain=-6:6,
	  			    xmin=-5, xmax=5,
				    ymin=-5, ymax=5,
				    xtick={-4, 0, 4}, ytick={-4, 0, 4},
				  	width=\linewidth
					]
					\addplot3[point meta={sqrt((y)^2 + (sin(deg(x)))^2)}, quiver={u=y, v=sin(deg(x)), scale arrows=0.2, every arrow/.append style={-{Stealth[scale=0.2+0.8*\pgfplotspointmetatransformed/1000]}}}, samples=16] (x,y,0);
				\end{axis}
			\end{tikzpicture}
			\subcaption{$\mathbf{F} :  (x, y) \mapsto \langle y, \sin x \rangle$}
		\end{subfigure}
		\hfil
		\begin{subfigure}{0.47\textwidth}
			\centering
			\begin{tikzpicture}
				\begin{axis}[
			  	    domain=-1:1,
				    xmin=-1, xmax=1,
				    ymin=-1, ymax=1,
				    zmin=-1, zmax=1,
				  	width=\linewidth,
					]
					\pgfplotsinvokeforeach{-1, -0.5, 0, 0.5, 1}{
						\addplot3[point meta={sqrt((x^2)^2)}, quiver={u=x^2, v=0, w=0, scale arrows=0.2, every arrow/.append style={-{Stealth[scale=0.2+0.8*\pgfplotspointmetatransformed/1000]}}}, samples=6] (x,y,#1);
					}
				\end{axis}
			\end{tikzpicture}
			\subcaption{$\mathbf{F} :  (x, y, z) \mapsto \langle x^2, 0, 0 \rangle$}
		\end{subfigure}
	\end{figure*}
	
	\par A beautiful observation from these plots, that is most vector fields can be seen as flowing fluid, which provide a beautiful and helpful interpretation about their behaviour.
	
	\par A place where vector field might appear is in the representation of gradient of multivariable scalar function. Lets consider the following example.
	\begin{example}
		Let $f: \mathbb{R}^2 \rightarrow \mathbb{R}$ be a scalar function act on $\mathbb{R}^2$, with $f(x, y) = x^2y - y^3$. Write out $\nabla f(x, y)$ and plot it.
	\end{example}
	\begin{solution}
		\begin{align*}
			\nabla f(x, y) = 
			\begin{bmatrix}
				\frac{\partial f}{\partial x} \\
				\frac{\partial f}{\partial y}
			\end{bmatrix}
			=
			\begin{bmatrix}
				2xy \\
				x^2 - 3y^2
			\end{bmatrix}
			\equiv
			2xy \;\mathbf{i} + (x^2 - 3y^2) \;\mathbf{j}
		\end{align*}
		
		The gradient $\nabla f(x, y)$ is a vector field act on $\mathbb{R}^2$, and its visualization 
		\begin{figure}[h]
			\centering
			\begin{tikzpicture}
				\begin{axis}[
				  	view={0}{90}, % for a view 'from above'
				  	domain=-7:7,
	  			    xmin=-7, xmax=7,
				    ymin=-7, ymax=7,
		  	        grid=both,
		  	        width=0.5\linewidth
					]
					\addplot3[point meta={sqrt((2*x*y)^2 + (x^2 - 3*y^2)^2)}, quiver={u=2*x*y, v=x^2 - 3*y^2, scale arrows=0.01, every arrow/.append style={-{Stealth[scale=0.2+0.8*\pgfplotspointmetatransformed/1000]}}}, samples=17] (x,y,0);
				\end{axis}
			\end{tikzpicture}
		\end{figure}
		
		This vector field hold information of the magnitude and direction of the gradient of $f(x,y)$ at any point $(x,y) \in \mathbb{R}^2$.
	\end{solution}
	
\end{document}