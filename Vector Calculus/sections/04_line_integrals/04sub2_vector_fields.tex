\documentclass[../../main.tex]{subfiles}
\graphicspath{{\subfix{../../media/}}}


\begin{document}
	
		\begin{definition}\label{def: scalar line integration}
		Line integration of a vector field $\mathbf{F}: \mathbb{R}^n \rightarrow \mathbb{R}^n$ along a smooth curve $C$
		\begin{equation}
			\int_C \mathbf{F} \cdot \mathbf{T} \; ds
		\end{equation}
		where $\mathbf{T}$ is the tangent vector to curve $C$
	\end{definition}
		
	\par As the definition of line integration in scalar field, the definition above is derived from the concept of Riemann summation. Let's take the following toy example to illustrate this.
	
	\begin{example}
		Given a vector field in $\mathbb{R}^2$ defined by $\mathbf{F}(x, y) = (x^2 - y^2 -12) \; \mathbf{i} + 2xy \; \mathbf{j}$ and a parametric curve $C$ defined by following equations 
		\begin{align*}
			x &= 6t \cos(5t)		&&y = 6t^3 - 6\\
		\end{align*}
		Derive the definition of line integration exploiting the concept of Riemann summation. Graph some Riemann approximation of the line integration of $\mathbf{F}(x, y)$ along curve $C$ for arbitrary $n$
	\end{example}
	\begin{solution}
		Riemann summation performed on $\mathbf{F}(x, y)$ along curve $C$ can be achieve by the following steps:
		\begin{enumerate}
			\item Divide the curve $C$ to $n$ small subarcs\footnote{small arcs} $\Delta s_1, \dots \Delta s_n$.
			\item For every subarc $i$, evaluate $\mathbf{F}(x,y)$ and the unit tangent vector to the curve  direction $\mathbf{T}(t)$ at some point\footnote{choice of $(x_i^*, y_i^*)$ will decide which Riemann sum we have. Either Right, Left, Midpoint Riemann sum} $t_i^* \equiv (x_i^*, y_i^*) \in \Delta s_i$.
			\item Compute $[\mathbf{F}(x_i^*, y_i^*) \, \cdot \, \mathbf{T}(t_i^*)] \; \Delta s_i$ for every subarc " i.e. the alignment between the vector field vector $\mathbf{F}(x_i^*, y_i^*)$ with the unit tangent vector $\mathbf{T}(t_i^*)$ of subarc $i$".
			\item Sum  $[\mathbf{F}(x_i^*, y_i^*) \, \cdot \, \mathbf{T}(x_i^*, y_i^*)] \; \Delta s_i$ over $n$ to approximate how much the vectors of $\mathbf{F}(x_i^*, y_i^*)$; along the path defined by $C$, align with $C$.
				\begin{equation*}
					\text{Work} \approx \sum_{i=1}^n [\mathbf{F}(x_i^*, y_i^*) \cdot \mathbf{T}(x_i^*, y_i^*)] \Delta s_i 
				\end{equation*}
			\item Riemann summation imply that $n$ goes to infinity. Hence the Riemann summation gives us the line integration\footnote{recall that $\int \equiv lim_{n \rightarrow \infty} \sum_{i=1}^n$} of $\mathbf{F}(x, y)$ along curve $C$
				\begin{equation*}
					\int_C \mathbf{F} \cdot \mathbf{T} \; ds= \lim_{n \rightarrow \infty} \sum_{i=1}^n [\mathbf{F}(x_i^*, y_i^*) \cdot \mathbf{T}(x_i^*, y_i^*)] \Delta s_i 
				\end{equation*}
		\end{enumerate}

		\begin{figure*}[h]
			\begin{subfigure}[c]{0.45\textwidth}
				\centering
				\begin{tikzpicture}[
						declare function={
							xs(\t) = 6*\t*cos(deg(5*\t));
							ys(\t) = 6*\t^3 - 6;
							xs_prime(\t) = 6*cos(deg(5*\t)) - 30*\t*sin(deg(5*\t));
							ys_p rime(\t) = 18*\t^2;
							}
						]
					\begin{axis}[
					  	view={0}{90}, % for a view 'from above'
				        colormap/viridis,
					  	domain=-7:7,
		  			    xmin=-7, xmax=7,
					    ymin=-7, ymax=7,
			  	        width=\linewidth
						]
						
						% function
					    \addplot3[contour filled={number = 50}, thick, opacity=0.8] {sqrt((\x^2 - \y^2 -12)^2 + (2*\x*\y)^2 )};

						\addplot3[white, point meta={sqrt((x^2-y^2-12)^2 + (2*x*y))^2)}, quiver={u=x^2-y^2-12, v=2*x*y, scale arrows=0.015, every arrow/.append style={-{Stealth[scale=0.2+0.8*\pgfplotspointmetatransformed/1000]}}}, samples=20] (x,y,0);	
						
						\addplot[red, thick, samples=20, parametricmarker, samples y=0, domain=0.5:1.1, variable=\t] ({xs(\t)}, {ys(\t)});
						
													
						\addplot[yellow, thick, samples=3, samples y=0, domain=0.799:0.904, variable=\t] ({xs(\t)}, {ys(\t)});
						\node at (axis cs:-1.0,0.2) [yellow, anchor=north east] {$\Delta s_i$};
													
						
					\end{axis}
				\end{tikzpicture}
			\end{subfigure}
			\hfil
			\begin{subfigure}[c]{0.45\textwidth}
				\centering
				\begin{tikzpicture}[
					declare function={
						xs(\t) = 6*\t*cos(deg(5*\t));
						ys(\t) = 6*\t^3 - 6;
						xs_prime(\t) = 6*cos(deg(5*\t)) - 30*\t*sin(deg(5*\t));
						ys_prime(\t) = 18*\t^2;
						}
					]
					\begin{axis}[
					  	view={0}{90}, % for a view 'from above'
				        colormap/viridis,
					  	domain=-7:7,
		  			    xmin=-7, xmax=7,
					    ymin=-7, ymax=7,
			  	        width=\linewidth
						]
						
						% function
					    \addplot3[contour filled={number = 50}, thick, opacity=0.8] {sqrt((\x^2 - \y^2 -12)^2 + (2*\x*\y)^2 )};
					    
						\addplot3[white, point meta={sqrt((x^2-y^2-12)^2 + (2*x*y))^2)}, quiver={u=x^2-y^2-12, v=2*x*y, scale arrows=0.015, every arrow/.append style={-{Stealth[scale=0.2+0.8*\pgfplotspointmetatransformed/1000]}}}, samples=20] (x,y,0);	
						
						\addplot[red, thick, samples=20, parametricmarker, samples y=0, domain=0.5:1.1, variable=\t] ({xs(\t)}, {ys(\t)});
						
						% unit tangent vector to curve 
						\pgfplotsinvokeforeach{0.5,0.78,0.91,1.0,1.09}{
							\addplot [->, thick,  black] coordinates {
								({xs(#1)}, {ys(#1)}) 
								({xs(#1) + xs_prime(#1)/(sqrt(xs_prime(#1)^2 + ys_prime(#1)^2))}, {ys(#1) + ys_prime(#1)/(sqrt(xs_prime(#1)^2 + ys_prime(#1)^2)})};
							
							\addplot [->, thick,  yellow] coordinates {
								({xs(#1)}, {ys(#1)}) 
								({xs(#1) + (xs(#1)^2 - ys(#1)^2 - 12)/(sqrt((xs(#1)^2 - ys(#1)^2 - 12)^2 + (2*xs(#1)*ys(#1))^2))}, {ys(#1) + (2*xs(#1)*ys(#1))/(sqrt((xs(#1)^2 - ys(#1)^2 - 12)^2 + (2*xs(#1)*ys(#1))^2))})};
							}
					\end{axis}
				\end{tikzpicture}
			\end{subfigure}
			\caption{Left: Illustrating $\Delta s_i$. Right: Riemann summation for small $n$ where vectors in yellow are $\mathbf{F}(x_i^*, y_i^*)$ and in black are $\mathbf{T}(x_i^*, y_i^*)$}
		\end{figure*}
	
		The plot above is the approximated line integration of $\mathbf{F}(x,y)$ along curve $C$ using Riemann sum.
	\end{solution}
	
	\par In context of parametric curves, it is more convenient\footnote{computationaly} to reformulate line integration formula  "presented in definition (\ref{def: scalar line integration})" in terms integration operator for the parametric variable "usually referred as $t$" rather than the arc length "usually referred as $s$" hence $ds \rightarrow dt$.
	
	\begin{definition}
		Line integration of a vector field $\mathbf{F}: \mathbb{R}^n \rightarrow \mathbb{R}^n$ along a smooth parametric curve $C$ defined by vector function $\mathbf{r}(t)$, $a \leq t \leq b$ 
		\begin{equation*}
			\int_C \mathbf{F} \cdot \; d\mathbf{r} = \int_a^b \mathbf{F}(\mathbf{r}(t)) \cdot \mathbf{r}^\prime(t) \; dt = \int_C \mathbf{F} \cdot \mathbf{T} \; ds
		\end{equation*}
	\end{definition}


	\begin{example}
		Evaluate $\int_C \mathbf{F} \cdot \, d\mathbf{r}$, where $C$ is a quarter-circle defined through $\mathbf{r}(t) = \cos t \; \mathbf{i} + \sin t \; \mathbf{j}$, $0 \leq t \leq \pi/2$ and $\mathbf{F}$ is a vector field $\mathbf{F}(x,y) = x^2 \; \mathbf{i} - xy \; \mathbf{j}$.
	\end{example}
	\begin{solution}
		From the parametric equation $x = \cos t$ and $y = \sin t$, hence we have
		\begin{equation*}
			\mathbf{F}(\mathbf{r}(t)) = \cos^2 t \; \mathbf{i} - \cos t \, \sin t \; \mathbf{j}		
		\end{equation*}
		and
		\begin{equation*}
			\mathbf{r}^\prime(t) = -\sin t \; \mathbf{i} + \cos t \; \mathbf{j}		
		\end{equation*}
		
		Therefore the line integration reads
		\begin{align*}
			\int_C \mathbf{F} \cdot \; d\mathbf{r} 	&= \int_0^{\pi/2} \left ( -\cos^2 t \sin t -\cos^2 t \sin t \right ) dt \\
								&= \int_0^{\pi/2} (-2\cos^2 t \sin t) dt \\
								&= \left\lbrack 2 \frac{\cos^3 t}{3}  \right\rbrack_0^{\pi/2} = - \frac{2}{3}
		\end{align*}
	\end{solution}
		
\end{document}