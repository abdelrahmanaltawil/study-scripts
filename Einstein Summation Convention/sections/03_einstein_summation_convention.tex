\documentclass[../main.tex]{subfiles}
\graphicspath{{\subfix{../media/}}}


\begin{document}

	\par In many fields, it is more convenient to represent vectors, matrices and tensors with identical notation, where lower indices "\textit{subscripts}" and upper indices "\textit{superscripts}" are appended to the variable representing the quantities of interest\footnote{this notation was introduced in Ricci calculus by Gregorio Ricci-Curbastro }. Even though, in mathematics \textit{subscripts} and \textit{superscripts} are interpreted differently, Einstein in his summation convention do not distinguish between them\footnote{that is, $x^2$ should be understood as the second component of $x$ not the square of $x$}.
		\begin{align*}
			& x_i \equiv x^i\\
 			& x_{ij} \equiv x^i_j \equiv x^{ij}
		\end{align*}
	\par Einstein introduced his summation convention to provide a simplified notations for sums (which is a vital operation in tensor calculus). He defined his convention based on the following rules.\\

	\noindent\textbf{\underline{Rules}}
	\begin{enumerate}
		\item If an index appears twice in a single term\footnote{monom $\hdots + \overbrace{z^n y^n x^n}^{\text{monom}} + \hdots$}. Then it implies summation over the two variables that carry this index. This index is referred as "\textit{dummy index}", where index that appears once is called "\textit{free/live index}".
			\begin{equation*}
				\hdots + a_i b_{jk} z_i + \hdots \equiv \hdots + \sum_{i=1}^{n} a_i b_{jk} z_i + \hdots
			\end{equation*}
			\underline{Hence, in this example $i$ is a dummy index where $j,k$ are live indices.}
		\item Number of free/live indices determine the order of the tensor (no live indices means scalar, one means vector, two means second order tensor and so on).
		\item An index should not appear more than twice in a single term. 
			\begin{equation*}
				\hdots + a_{\color{red}i} b_{{\color{red}i}k} z_{\color{red}i} + \hdots \rightarrow \textbf{not valid}
			\end{equation*}
			\underline{One of possible exceptions is $a_i (b_i + c_i)$ since we add $b_i, \; c_i$ first.}
	\end{enumerate}
		
\end{document}